\documentclass{article}
\usepackage[utf8]{inputenc}
\usepackage{listings}
\usepackage{graphicx}
\usepackage{float}
\usepackage{xcolor}
\usepackage{geometry}
\usepackage{CJKutf8}
\usepackage{amsmath}
\usepackage{amssymb}

\geometry{a4paper,scale=0.8}
\lstset{
    basicstyle          =   \sffamily,        
    keywordstyle        =   \bfseries,         
    commentstyle        =   \rmfamily\itshape, 
    stringstyle         =   \ttfamily, 
    flexiblecolumns,               
    numbers             =   left,  
    showspaces          =   false, 
    showstringspaces    =   false,
    captionpos          =   t,     
    frame               =   lrtb, 
}

\lstdefinestyle{Python}{
    language        =   Python, % 语言选Python
    basicstyle      =   \zihao{-5}\ttfamily,
    numberstyle     =   \zihao{-5}\ttfamily,
    keywordstyle    =   \color{blue},
    keywordstyle    =   [2] \color{teal},
    stringstyle     =   \color{magenta},
    commentstyle    =   \color{red}\ttfamily,
    breaklines      =   true,  
    columns         =   fixed,  
    basewidth       =   0.5em,
}

\title{\bf\Large  概率论与数理统计 第11次作业}
%%%%%%%%%%%%%%%%%%%%%%%%%%%%%%%%%%%%%%
%% DON'T forget to change this part %%
\author{\bf Name: 宋昊原 \qquad Student ID: 2022010755}
%%%%%%%%%%%%%%%%%%%%%%%%%%%%%%%%%%%%%%

\begin{document}
\begin{CJK}{UTF8}{gbsn}
\maketitle
\section{合格率区间估计}
根据中心极限定理,近似地有
$$ \frac{p^{*}-p}{\sqrt{\frac{p(1-p)}{n}}}\sim N(0,1)$$
其中
$$ p^{*}=0.6$$
$$ n=100$$
以$p^{*}$估计分母中的$p$,可以得到$1-\alpha$置信的区间估计:
$$(p^{*}-\sqrt{\frac{p^{*}(1-p^{*})}{n}}\Phi^{-1}(1-\frac{\alpha}{2}),p^{*}+\sqrt{\frac{p^{*}(1-p^{*})}{n}}\Phi^{-1}(1-\frac{\alpha}{2}))$$
代入
$$\alpha = 0.05$$
解得0.95置信区间为
$$ (0.504,0.696)$$
\section{幂函数分布的估计}
\subsection{}
似然函数
$$ L(\theta)=\prod\limits_{i=1}^{n}\theta X_{i}^{\theta-1}=\theta^{n}(\prod\limits_{i=1}^{n}X_{i})^{\theta-1}$$
对数似然函数
$$ l(\theta)=\ln L(\theta)=n\ln\theta+(\theta-1)\sum\limits_{i=1}^{n}\ln X_{i}$$
故
$$ l'(\theta)=\frac{n}{\theta}+\sum\limits_{i=1}^{n}\ln X_{i}$$
故
$$ \theta^{*}=-\frac{n}{\sum\limits_{i=1}^{n}\ln X_{i}}$$
这是最大值点.
\\代入数据有
$$ \theta^{*}=0.7983$$
下面求Fisher信息量
$$ I(\theta)=E((\frac{d}{d\theta}(\ln\theta+(\theta-1)\ln x))^{2})$$
$$ =\int_{0}^{1}(\frac{1}{\theta}+\ln x)^{2}\theta x^{\theta-1}dx$$
$$ =\int_{0}^{1}\frac{x^{\theta-1}}{\theta}dx+2\int_{0}^{1}x^{\theta-1}\ln xdx+\int_{0}^{1}x^{\theta-1}(\ln x)^{2}dx$$
$$ =\frac{1}{\theta^{2}}-\frac{2}{\theta^{2}}+\frac{2}{\theta^{2}}=\frac{1}{\theta^{2}}$$
标准差的估计值
$$ \sigma_{n}=\sqrt{\frac{1}{nI(\theta)}}\approx\sqrt{\frac{1}{nI(\theta^{*})}}=\frac{\theta^{*}}{\sqrt{n}}=0.3992$$
\section{正态总体的估计}
\subsection{}
对数似然函数
$$ l(\sigma)=-n\ln(\sqrt{2\pi}\sigma)-\sum\limits_{i=1}^{n}\frac{(X_{i}-\mu)^{2}}{2\sigma^{2}} $$
令
$$ l'(\sigma)=-\frac{n}{\sigma}+\frac{\sum\limits_{i=1}^{n}(X_{i}-\mu)^{2}}{\sigma^{3}}=0$$
解得
$$ \sigma^{*}=\sqrt{\frac{\sum\limits_{i=1}^{n}(X_{i}-\mu)^{2}}{n}}$$
\subsection{}
计算Fisher信息量
$$ I(\sigma)=E((\frac{\partial\ln f}{\partial\sigma})^{2})=E((-\frac{1}{\sigma}+\frac{(x-\mu)^{2}}{\sigma^{3}})^{2})$$
$$ =\int_{-\infty}^{+\infty}\frac{1}{\sqrt{2\pi}\sigma^{3}}\exp(-\frac{(x-\mu)^{2}}{2\sigma^{2}})dx-2\int_{-\infty}^{+\infty}\frac{(x-\mu)^{2}}{\sqrt{2\pi}\sigma^{5}}\exp(-\frac{(x-\mu)^{2}}{2\sigma^{2}})dx+\int_{-\infty}^{+\infty}\frac{(x-\mu)^{4}}{\sqrt{2\pi}\sigma^{7}}\exp(-\frac{(x-\mu)^{2}}{2\sigma^{2}})dx$$
$$ =\frac{1}{\sigma^{2}}-\frac{2Var(N(0,1))}{\sigma^{2}}+\frac{Kurt(N(0,1))}{\sigma^{2}}=\frac{2}{\sigma^{2}}$$
故$\sigma^{*}$标准误差的估计
$$ \sqrt{\frac{1}{nI(\sigma^{*})}}=\frac{\sigma}{\sqrt{2n}}$$
由于近似地有
$$ \frac{\sigma^{*}-\sigma}{\frac{\sigma}{\sqrt{2n}}}\sim N(0,1)$$
可以给出其$1-\alpha$置信区间
$$ (\sigma^{*}-\frac{\sigma^{*}}{\sqrt{2n}}\Phi^{-1}(1-\frac{\alpha}{2}),\sigma^{*}+\frac{\sigma^{*}}{\sqrt{2n}}\Phi^{-1}(1-\frac{\alpha}{2})) $$
其中
$$ \Phi^{-1} $$
为标准正态累积分布函数的反函数.
\section{方差不同样本的差值估计}
近似地
$$ \frac{(\bar{X}-\bar{Y})-(\mu_{1}-\mu_{2})}{\sqrt{\frac{\sigma_{1}^{2}}{n_{1}}+\frac{\sigma_{2}^{2}}{n_{2}}}}\sim N(0,1)$$
以$S_{1}^{2},S_{2}^{2}$估计$\sigma_{1}^{2},\sigma_{2}^{2}$,有
$$ \frac{(\bar{X}-\bar{Y})-(\mu_{1}-\mu_{2})}{\sqrt{\frac{S_{1}^{2}}{n_{1}}+\frac{S_{2}^{2}}{n_{2}}}}\sim N(0,1)$$
于是$1-\alpha$置信区间估计为
$$ (\bar{X}-\bar{Y}-\sqrt{\frac{S_{1}^{2}}{n_{1}}+\frac{S_{2}^{2}}{n_{2}}}\Phi^{-1}(1-\frac{\alpha}{2}),\bar{X}-\bar{Y}+\sqrt{\frac{S_{1}^{2}}{n_{1}}+\frac{S_{2}^{2}}{n_{2}}}\Phi^{-1}(1-\frac{\alpha}{2})) $$
代入数据计算得
$$ (-3.14,\ -0.90) $$
\section{后验约会}
$\theta$的先验分布为
$$ f_{\Theta}(\theta)=1,\ \theta\in(0,1)$$
条件分布
$$ f_{X|\Theta}(x|\theta)=\frac{1}{\theta},\ x\in(0,\theta)$$
于是$X$的边际分布
$$ f_{X}(x)=\int_{x}^{1}\frac{1}{\theta}d\theta=-\ln x $$
于是后验分布
$$ f_{\Theta|X}(\theta|x)=\frac{1\times\frac{1}{\theta}}{-\ln x}=-\frac{1}{\theta\ln x},\ \theta\in(x,1)$$
\section{硬币的后验估计}
后验分布密度函数为
$$ f(\theta)=\frac{\theta^{x}(1-\theta)^{n-x}}{\int_{0}^{1}\theta^{x}(1-\theta)^{n-x}d\theta}=\frac{\theta^{x}(1-\theta)^{n-x}}{B(x+1,n-x+1)}$$
求对数密度函数的极值,令
$$ \frac{d\ln f(\theta)}{d\theta}=\frac{x}{\theta}-\frac{n-x}{1-\theta}=0$$
解得
$$ \hat{\theta}=\frac{x}{n}$$
在此题中为
$$ \hat{\theta}=\frac{13}{20}=0.65 $$
事实上此后验分布密度函数与似然函数成正比,因此与极大似然估计得到的结论相同.
\section{正态分布的后验估计}
首先求出其后验分布.\\
先验分布
$$ f_{M}(\mu)=\frac{1}{\sqrt{2\pi}\sigma_{0}}\exp(-\frac{(\mu-\mu_{0})^{2}}{2\sigma_{0}^{2}})$$
条件分布
$$ f_{X_{1},...,X_{n}|M}(x_{1},...,x_{n}|\mu)=\prod\limits_{i=1}^{n}\frac{1}{\sqrt{2\pi}\sigma}\exp(-\frac{(x_{i}-\mu)^{2}}{2\sigma^{2}})=\frac{1}{(\sqrt{2\pi}\sigma)^{n}}\exp(-\frac{\sum\limits_{i=1}^{n}(x_{i}-\mu)^{2}}{2\sigma^{2}})$$
边际分布
$$ f_{X_{1},...,X_{n}}(x_{1},...,x_{n})=\int_{-\infty}^{+\infty}\frac{1}{\sqrt{2\pi}\sigma_{0}(\sqrt{2\pi}\sigma)^{n}}\exp(-\frac{\sum\limits_{i=1}^{n}(x_{i}-\mu)^{2}}{2\sigma^{2}}-\frac{(\mu-\mu_{0})^{2}}{2\sigma_{0}^{2}})d\mu=I(\mathbf{x})$$
与$\mu$无关.\\
故后验分布为
$$ g(\mu)=\frac{\frac{1}{\sqrt{2\pi}\sigma_{0}(\sqrt{2\pi}\sigma)^{n}}\exp(-\frac{\sum\limits_{i=1}^{n}(x_{i}-\mu)^{2}}{2\sigma^{2}}-\frac{(\mu-\mu_{0})^{2}}{2\sigma_{0}^{2}})}{I(\mathbf{x})}$$
\subsection{最大后验估计}
$$ \frac{\partial\ln g(\mu)}{\partial\mu}=-\frac{\sum\limits_{i=1}^{n}(\mu-x_{i})}{\sigma^{2}}-\frac{(\mu-\mu_{0})}{\sigma_{0}^{2}}$$
令上述导数为0,则
$$ \hat{\mu}_{1}=\frac{n\sigma_{0}^{2}\bar{X}+\sigma^{2}\mu_{0}}{n\sigma_{0}^{2}+\sigma^{2}}$$
\subsection{后验均值估计}
$$ \hat{\mu}_{2}=\int_{-\infty}^{+\infty}g(\mu)d\mu=\int_{-\infty}^{+\infty}\frac{\frac{\mu}{\sqrt{2\pi}\sigma_{0}(\sqrt{2\pi}\sigma)^{n}}\exp(-\frac{\sum\limits_{i=1}^{n}(x_{i}-\mu)^{2}}{2\sigma^{2}}-\frac{(\mu-\mu_{0})^{2}}{2\sigma_{0}^{2}})}{I(\mathbf{x})}d\mu$$
$$ =\int_{-\infty}^{+\infty}\frac{\frac{\mu}{\sqrt{2\pi}\sigma_{0}(\sqrt{2\pi}\sigma)^{n}}\exp(-(\frac{n}{2\sigma^{2}}+\frac{1}{2\sigma_{0}^{2}})\mu^{2}+(\frac{n\bar{X}}{\sigma^{2}}+\frac{\mu_{0}}{\sigma_{0}^{2}})\mu-C_{1}(\mathbf{X}))}{I(\mathbf{X})}d\mu$$
其中$C_{1}(\mathbf{X})$与$\mu$无关.
\\
\\令
$$ \hat{\mu}_{1}=\frac{n\sigma_{0}^{2}\bar{X}+\sigma^{2}\mu_{0}}{n\sigma_{0}^{2}+\sigma^{2}}$$
$$ \gamma = \frac{n}{\sigma^{2}}+\frac{1}{\sigma_{0}^{2}}$$
则
$$ \hat{\mu}_{2}=\frac{1}{\sqrt{2\pi}\sigma_{0}(\sqrt{2\pi}\sigma)^{n}I(\mathbf{X})}\int_{-\infty}^{+\infty}\mu\exp(-\frac{\gamma}{2}(\mu-\hat{\mu}_{1})^{2}+C_{2}(\mathbf{X}))d\mu$$
其中$C_{2}(\mathbf{X})$为配方的剩余项,与$\mu$无关.
\\\\此时将$d\mu$调整至与指数函数内的部分相同,得
$$ =\frac{1}{\sqrt{2\pi}\sigma_{0}(\sqrt{2\pi}\sigma)^{n}I(\mathbf{X})}\frac{1}{\gamma}\int_{-\infty}^{+\infty}\exp(-\gamma(\mu-\hat{\mu}_{1})^{2}+C_{2}(\mathbf{X}))d(-\gamma(\mu-\hat{\mu}_{1})^{2}+C_{2}(\mathbf{X}))$$
$$+\frac{1}{\sqrt{2\pi}\sigma_{0}(\sqrt{2\pi}\sigma)^{n}I(\mathbf{X})}\int_{-\infty}^{+\infty}\hat{\mu}_{1}\exp(-\frac{\gamma}{2}(\mu-\hat{\mu}_{1})^{2}+C_{2}(\mathbf{X}))d\mu$$
$$ =0+\frac{\hat{\mu}_{1}I(\mathbf{X})}{I(\mathbf{X})}=\hat{\mu}_{1}$$
故后验均值估计和最大后验估计得到同一结果:
$$ \hat{\mu}_{1}=\frac{n\sigma_{0}^{2}\bar{X}+\sigma^{2}\mu_{0}}{n\sigma_{0}^{2}+\sigma^{2}}$$
\section{几何分布的贝叶斯估计}
\subsection{后验分布}
先验分布
$$ f_{\Theta}(\theta)=1,\ \theta\in(0,1)$$
用$\mathbf{X}$表示样本向量,则条件分布
$$ f_{\mathbf{X}|\Theta}(\mathbf{x}|\theta)=\theta^{n}(1-\theta)^{\sum\limits_{i=1}^{n}x_{i}}=\theta^{3}(1-\theta)^{10}$$
边际分布
$$ f_{\mathbf{X}}(\mathbf{x})=\int_{0}^{1}\theta^{3}(1-\theta)^{10}d\theta=\int_{0}^{1}(1-\theta)^{3}\theta^{10}d\theta$$
$$ =\int_{0}^{1}(\theta^{10}-3\theta^{11}+3\theta^{12}-\theta^{13})d\theta=\frac{1}{11}-\frac{3}{12}+\frac{3}{13}-\frac{1}{14}=\frac{1}{4004}$$
故后验分布
$$ g(\theta)=4004\theta^{3}(1-\theta)^{10},\ \theta\in(0,1)$$
\subsection{后验均值分布}
$$\hat{\theta}=\int_{0}^{1}4004\theta^{4}(1-\theta)^{10}d\theta=\int_{0}^{1}4004(1-\theta)^{4}\theta^{10}d\theta$$
$$=4004(\frac{1}{11}-\frac{4}{12}+\frac{6}{13}-\frac{4}{14}+\frac{1}{15})=\frac{4}{15}$$
\section{Bayes区间估计}
\subsection{}
根据第7题,后验分布为
$$ g(\mu)=\frac{\frac{1}{\sqrt{2\pi}\sigma_{0}(\sqrt{2\pi}\sigma)^{n}}\exp(-\frac{\sum\limits_{i=1}^{n}(x_{i}-\mu)^{2}}{2\sigma^{2}}-\frac{(\mu-\mu_{0})^{2}}{2\sigma_{0}^{2}})}{I(\mathbf{X})}$$
$$ =\frac{\frac{1}{\sqrt{2\pi}\sigma_{0}(\sqrt{2\pi}\sigma)^{n}}\exp(-\frac{\gamma}{2}(\mu-\hat{\mu}_{1})^{2}+C_{2}(\mathbf{X}))}{I(\mathbf{X})}$$
这符合正态分布的形式,于是后验分布为
$$ \mu \sim N(\hat{\mu}_{1},\frac{1}{\gamma})$$
其中
$$ \hat{\mu}_{1}=\frac{n\sigma_{0}^{2}\bar{X}+\sigma^{2}\mu_{0}}{n\sigma_{0}^{2}+\sigma^{2}}$$
$$ \tau^{2}:=\frac{1}{\gamma} = \frac{1}{\frac{n}{\sigma^{2}}+\frac{1}{\sigma_{0}^{2}}}$$
于是
$$ a=\hat{\mu}_{1}-\tau\Phi^{-1}(1-\frac{\alpha}{2})=\frac{n\sigma_{0}^{2}\bar{X}+\sigma^{2}\mu_{0}}{n\sigma_{0}^{2}+\sigma^{2}}-\sqrt{\frac{1}{\frac{n}{\sigma^{2}}+\frac{1}{\sigma_{0}^{2}}}}\Phi^{-1}(1-\frac{\alpha}{2})$$
$$ b=\hat{\mu}_{1}+\tau\Phi^{-1}(1-\frac{\alpha}{2})=\frac{n\sigma_{0}^{2}\bar{X}+\sigma^{2}\mu_{0}}{n\sigma_{0}^{2}+\sigma^{2}}+\sqrt{\frac{1}{\frac{n}{\sigma^{2}}+\frac{1}{\sigma_{0}^{2}}}}\Phi^{-1}(1-\frac{\alpha}{2})$$
其中$\Phi^{-1}$为标准正态累积分布函数的反函数.
\subsection{}
$\sigma_{0}\to\infty$时,有
$$ a=\bar{X}-\frac{\sigma}{\sqrt{n}}\Phi^{-1}(1-\frac{\alpha}{2})$$
$$ b=\bar{X}+\frac{\sigma}{\sqrt{n}}\Phi^{-1}(1-\frac{\alpha}{2})$$
这与经典方法得到的结论一致.
\\\\
事实上,$\sigma_{0}\to\infty$说明先验分布的方差为无穷大,即先验分布实际上类似于一个“$\mathbb{R}$上的均匀分布”,这相当于我们对$\mu$一无所知的情况下得到的结论.
\subsection{}
条件分布
$$ f_{\mathbf{X}|M}(\mathbf{x}|\mu)=\frac{1}{(\sqrt{2\pi}\sigma)^{n}}\exp(-\frac{\sum\limits_{i=1}^{n}(x_{i}-\mu)^{2}}{2\sigma^{2}})$$
边际分布
$$ f_{\mathbf{X}}(\mathbf{x})=\int_{-\infty}^{+\infty}f(\mu)\frac{1}{(\sqrt{2\pi}\sigma)^{n}}\exp(-\frac{\sum\limits_{i=1}^{n}(x_{i}-\mu)^{2}}{2\sigma^{2}})d\mu$$
故后验分布
$$ g(\mu)=\frac{f(\mu)\frac{1}{(\sqrt{2\pi}\sigma)^{n}}\exp(-\frac{\sum\limits_{i=1}^{n}(x_{i}-\mu)^{2}}{2\sigma^{2}})}{\int_{-\infty}^{+\infty}f(\mu)\frac{1}{(\sqrt{2\pi}\sigma)^{n}}\exp(-\frac{\sum\limits_{i=1}^{n}(x_{i}-\mu)^{2}}{2\sigma^{2}})d\mu}$$
$f(\mu)$是“常数”,故
$$ g(\mu|\mathbf{X})=\frac{\exp(-\frac{\sum\limits_{i=1}^{n}(X_{i}-\mu)^{2}}{2\sigma^{2}})}{\int_{-\infty}^{+\infty}\exp(-\frac{\sum\limits_{i=1}^{n}(X_{i}-\mu)^{2}}{2\sigma^{2}})d\mu}=\frac{\exp(-\frac{n\mu^{2}-2n\bar{X}\mu+C(\mathbf{X})}{2\sigma^{2}})}{\int_{-\infty}^{+\infty}\exp(-\frac{n\mu^{2}-2n\bar{X}\mu+C(\mathbf{X})}{2\sigma^{2}})d\mu}$$
其中$C(\mathbf{X})$与$\mu$无关,这符合正态分布形态,故后验分布为
$$ \mu\sim N(\bar{X},\frac{\sigma^{2}}{n})$$
最大后验区间和等尾可信区间均为
$$(\bar{X}-\frac{\sigma}{\sqrt{n}}\Phi^{-1}(1-\frac{\alpha}{2}),\bar{X}+\frac{\sigma}{\sqrt{n}}\Phi^{-1}(1-\frac{\alpha}{2}))$$
这与第二问结论相同,因为认为先验分布是$f(\mu)\propto 1$实际上等价于对$\mu$一无所知.
\section{计算机实验:自助法Bootstrap(续)}
\subsection{$\theta$的区间估计}
在第10次作业计算机实验中的模拟程序中,求出$\hat{\theta}^{*}$,并求其0.025和0.975分位数,得到区间估计为
$$ (123.000526,\ 184.629933) $$
\subsection{区间估计的其他方式}
可以使用课上提到的极大似然区间估计的方法,借助Fisher信息量$I(\theta)$来估计$Var(\theta)$.
\end{CJK}
\end{document}
\documentclass{article}
\usepackage[utf8]{inputenc}
\usepackage{listings}
\usepackage{graphicx}
\usepackage{float}
\usepackage{xcolor}
\usepackage{geometry}
\usepackage{CJKutf8}
\usepackage{amsmath}
\usepackage{amssymb}

\geometry{a4paper,scale=0.8}
\lstset{
    basicstyle          =   \sffamily,        
    keywordstyle        =   \bfseries,         
    commentstyle        =   \rmfamily\itshape, 
    stringstyle         =   \ttfamily, 
    flexiblecolumns,               
    numbers             =   left,  
    showspaces          =   false, 
    showstringspaces    =   false,
    captionpos          =   t,     
    frame               =   lrtb, 
}

\lstdefinestyle{Python}{
    language        =   Python, % 语言选Python
    basicstyle      =   \zihao{-5}\ttfamily,
    numberstyle     =   \zihao{-5}\ttfamily,
    keywordstyle    =   \color{blue},
    keywordstyle    =   [2] \color{teal},
    stringstyle     =   \color{magenta},
    commentstyle    =   \color{red}\ttfamily,
    breaklines      =   true,  
    columns         =   fixed,  
    basewidth       =   0.5em,
}

\title{\bf\Large  概率论与数理统计 第14次作业}
%%%%%%%%%%%%%%%%%%%%%%%%%%%%%%%%%%%%%%
%% DON'T forget to change this part %%
\author{\bf Name: 宋昊原 \qquad Student ID: 2022010755}
%%%%%%%%%%%%%%%%%%%%%%%%%%%%%%%%%%%%%%

\begin{document}
\begin{CJK}{UTF8}{gbsn}
\maketitle
\section{正态分布似然比检验}
似然函数
$$ L(\mu)=\frac{1}{(\sqrt{2\pi}\sigma^{2})^{n}}\exp(-\sum\limits_{i=1}^{n}\frac{(X_{i}-\mu)^{2}}{2\sigma^{2}}) $$
极大似然估计值
$$ \mu^{*}=\bar{X}$$
故似然比
$$ \Lambda=\exp(\sum\limits_{i=1}^{n}\frac{(X_{i}-\bar{X})^{2}}{2\sigma^{2}}-\sum\limits_{i=1}^{n}\frac{(X_{i}-\mu_{0})^{2}}{2\sigma^{2}})$$
$$ =\exp(\frac{\sum\limits_{i=1}^{n}(\bar{X}^{2}-\mu_{0}^{2}-2X_{i}\bar{X}+2X_{i}\mu_{0})}{2\sigma^{2}})$$
$$ =\exp(-\frac{(\bar{X}-\mu_{0})^{2}}{2\sigma^{2}})$$
于是,$n$很大时,近似地有
$$ -2\ln\Lambda = \frac{n(\bar{X}-\mu_{0})^{2}}{\sigma^{2}}\sim\chi^{2}(1)$$
即
$$ \frac{\bar{X}-\mu_{0}}{\sigma/\sqrt{n}}\sim{\rm N}(0,1)$$
拒绝域形状为
$$ \Lambda\leq\lambda_{0} $$
这等价于
$$ \vert\frac{\bar{X}-\mu_{0}}{\sigma/\sqrt{n}}\vert\geq c $$
其中
$$ c=Z_{\frac{\alpha}{2}}=Z_{0.025} $$
与Z检验得到同样的结果.
\section{Mendel似然比检验}
似然比为
$$\Lambda=\frac{\prod\limits_{i=1}^{4}p_{i}^{O_{i}}}{\prod\limits_{i=1}^{4}(\frac{O_{i}}{n})^{O_{i}}}=\prod\limits_{i=1}^{4}(\frac{E_{i}}{O_{i}})^{O_{i}}$$
故
$$ -2\ln\Lambda=2\sum\limits_{i=1}^{4}O_{i}\ln\frac{O_{i}}{E_{i}}$$
其中$O_{i}$与$E_{i}$较接近,在$E_{i}$处作Taylor展开,得
$$ -2\ln\Lambda=2\sum\limits_{i=1}^{4}(O_{i}-E_{i}+\frac{(O_{i}-E_{i})^{2}}{2E_{i}})=\sigma_{i=1}^{4}\frac{(O_{i}-E_{i})^{2}}{E_{i}}=\chi^{2}$$
近似地有
$$ \chi^{2}\sim\chi^{2}(3)$$
这与Pearson的$\chi^{2}$检验吻合.
\\\\代入计算,P值为0.9254,没有理由拒绝Mendel理论.
\section{图像制作}
假设两总体都是正态分布,则若方差相等,有
$$ \frac{(n_{1}-1)S_{1}^{2}}{(n_{2}-1)S_{2}^{2}}\sim{\rm F}(n_{1}-1,n_{2}-1)$$
拒绝域为
$$ \frac{(n_{1}-1)S_{1}^{2}}{(n_{2}-1)S_{2}^{2}}\geq F_{\frac{\alpha}{2}}(n_{1}-1,n_{2}-1)\ \lor\ \frac{(n_{1}-1)S_{1}^{2}}{(n_{2}-1)S_{2}^{2}}\leq F_{1-\frac{\alpha}{2}}(n_{1}-1,n_{2}-1)$$
计算得
$$ \frac{(n_{1}-1)S_{1}^{2}}{(n_{2}-1)S_{2}^{2}}=0.0575 < F_{0.025}(6,4)=0.1087$$
故可以拒绝原假设,两总体方差不等.
\section{机床加工}
\subsection{}
$$ H_{0}:\ \mu_{1}=\mu_{2} $$
$$ H_{1}:\ \mu_{1}\neq\mu_{2} $$
$H_{0}$成立时
$$ \frac{\bar{X}-\bar{Y}}{\sigma\sqrt{\frac{1}{n_{1}}+\frac{1}{n_{2}}}}\sim{\rm N}(0,1)$$
而
$$ \frac{(n_{1}-1)S_{1}^{2}+(n_{2}-1)S_{2}^{2}}{\sigma^{2}}\sim\chi^{2}(n_{1}+n_{2}-2)$$
于是
$$ \frac{\bar{X}-\bar{Y}}{\sqrt{(\frac{1}{n_{1}}+\frac{1}{n_{2}})((n_{1}-1)S_{1}^{2}+(n_{2}-1)S_{2}^{2})}}\sim{\rm t}(n_{1}+n_{2}-2)$$
左边的统计量计算得
$$ 0.7845 $$
而t(18)分布的0.975分位数为
$$ 2.1009 $$
故不能拒绝原假设,生产稳定.
\subsection{}
$$ H_{0}:\ \sigma_{1}^{2}=\sigma_{2}^{2}$$
$$ H_{1}:\ \sigma_{1}^{2}\neq\sigma_{2}^{2}$$
$H_{0}$成立时有
$$ \frac{(n_{1}-1)S_{1}^{2}}{(n_{2}-1)S_{2}^{2}}\sim{\rm F}(n_{1}-1,n_{2}-1)$$
左边统计量计算得
$$ 1.955 $$
而F(9,9)分布的(0.025,\ 0.975)分位数区间为
$$ (0.248,\ 4.026) $$
故没有理由拒绝$H_{0}$,支持方差相同的假设.
\section{地理是数学老师教的}
设
$$ (X,Y)\sim{\rm N}(\mu_{1},\mu_{2},\sigma_{1}^{2},\sigma_{2}^{2},\rho) $$
则假设
$$ H_{0}:\ \mu_{1}=\mu_{2} $$
$$ H_{1}:\ \mu_{1}>\mu_{2} $$
由于$X$和$Y$不一定独立,考察$X-Y$,当$H_{0}$成立时,其服从均值为0,方差未知的正态分布,于是
$$ \frac{\bar{X}-\bar{Y}}{S/\sqrt{n}}\sim{\rm t}(n-1)$$
其中$S$为$X-Y$的样本标准差.
\\\\左边的统计量计算得3.559,而t(9)分布的0.95分位数为1.833,故有理由拒绝$H_{0}$,两组成绩有显著差异.
\section{Bayes假设检验}
\subsection{}
后验概率之比
$$ \frac{{\rm P}(H_{0}\vert x)}{{\rm P}(H_{1}\vert x)}=1\times\frac{0.5^{x}0.5^{10-x}}{0.7^{x}0.3^{10-x}}=(\frac{5}{7})^{x}(\frac{5}{3})^{10-x}=(\frac{3}{7})^{x}(\frac{5}{3})^{10}$$
于是,拒绝$H_{0}$等价于上式小于1,即
$$ X>10\log_{\frac{3}{7}}(\frac{3}{5})=10\frac{\ln3-\ln5}{\ln3-\ln7}=6.03$$
即
$$ X\geq 7 $$
接受$H_{0}$等价于上式大于1,即
$$ X\leq 6 $$
于是第一类错误的概率为
$$ {\rm P}(X\geq 7\ \vert\ p=0.5)=0.376$$
第二类错误的概率为
$$ {\rm P}(X\leq 6\ \vert\ p=0.7)=0.150$$
\subsection{}
后验概率之比
$$ \frac{{\rm P}(H_{0}\vert x)}{{\rm P}(H_{1}\vert x)}=10\times\frac{0.5^{x}0.5^{10-x}}{0.7^{x}0.3^{10-x}}=10(\frac{5}{7})^{x}(\frac{5}{3})^{10-x}=10(\frac{3}{7})^{x}(\frac{5}{3})^{10}$$
于是,拒绝$H_{0}$等价于上式小于1,即
$$ X>\log_{\frac{3}{7}}(\frac{1}{10}(\frac{3}{5})^{10})=\frac{10\ln3-11\ln5-\ln2}{\ln3-\ln7}=8.75$$
即
$$ X\geq 9 $$
接受$H_{0}$等价于上式大于1,即
$$ X\leq 8 $$
于是第一类错误的概率为
$$ {\rm P}(X\geq 9\ \vert\ p=0.5)=0.054$$
第二类错误的概率为
$$ {\rm P}(X\leq 8\ \vert\ p=0.7)=0.617$$
\section{单尾t分布假设检验}
\subsection{检验结果}
设$X$是任意一个总体,当$H_{0}$成立时
$$\frac{\bar{X}-\mu_{0}}{S/\sqrt{n}}\sim{\rm t}(n-1) $$
故拒绝$H_{0}$的条件为
$$ \bar{X}<\mu_{0}-\frac{S}{\sqrt{n}}t_{\alpha}$$
对总体$A$,计算得
$$ \bar{A}=99.6 $$
$$ \mu_{0}-\frac{S_{A}}{\sqrt{n_{A}}}t_{\alpha}=99.7 $$
故对总体$A$拒绝原假设.\\\\
对总体$B$,计算得
$$ \bar{B}=-150 $$
$$ \mu_{0}-\frac{S_{B}}{\sqrt{n_{B}}}t_{\alpha}=-215.7 $$
故对总体$B$不拒绝原假设.
\subsection{看法}
统计显著和事实显著有很大区别.\ \ 统计显著性如何受到样本本身的波动情况和样本容量影响很大,样本容量越小,样本的标准差越大,拒绝的临界值就会越极端,从而统计显著会包容一些主观感受上非常极端的样本,例如上面的$B$.
\end{CJK}
\end{document}